\documentclass[12pt]{amsart}
\usepackage{mathrsfs}
\usepackage{amsmath}
\usepackage{amssymb}
\usepackage{amsfonts}
\usepackage{amsopn}
\usepackage{amsthm}
\usepackage{latexsym}
\usepackage[all]{xy}
\usepackage{enumerate}
\usepackage{geometry}
%\usepackage{biblatex}
%\usepackage{hyperref}
%\usepackage[autostyle]{csquotes}
\usepackage{fancyhdr}
\usepackage{graphicx}
\usepackage{wrapfig}
\usepackage{float}

\usepackage[
    backend=biber,
style=alphabetic,
sorting=nyt,
bibencoding=utf8,
   % style=authoryear-icomp,
    %sortlocale=de_DE,
    %natbib=true,
    %author=true,
%style=verbose,
%journal=true,
%url=true, 
%    doi=false,
%    eprint=true
]{biblatex}
\addbibresource{STABLEreferencesX.bib}

\usepackage[]{hyperref}
\hypersetup{
    colorlinks=true,
}


\newtheorem{thm}{Theorem}
\newtheorem{lem}[thm]{Lemma}
\newtheorem{prop}[thm]{Proposition}
\newtheorem*{prob}{Problem}
\newtheorem{cor}[thm]{Corollary}
\newtheorem{question}[thm]{Question}
\newtheorem*{hyp}{Hypothesis}
\newtheorem*{thmm}{Theorem}
\theoremstyle{definition}
\newtheorem{dfn}[thm]{Definition}
\newtheorem{exx}[thm]{Example}
\theoremstyle{remark}
\newtheorem{rem}[thm]{Remark}
\newcommand{\fh}{\mathfrak{h}}
\newcommand{\bn}{\mathbf{n}}
\newcommand{\bC}{\mathbb{C}}
\newcommand{\bG}{\mathbb{G}}
\newcommand{\bR}{\mathbb{R}}
\newcommand{\fB}{\mathfrak{B}}
\newcommand{\bZ}{\mathbb{Z}}
\newcommand{\bQ}{\mathbb{Q}}
\newcommand{\bq}{\bar{Q}[t]}
\newcommand{\bb}{\bullet}
\newcommand{\del}{\partial}
\newcommand{\sB}{\mathscr{B}}
\newcommand{\sE}{\mathscr{E}}
\newcommand{\mR}{\bR^\times_{>0}}
\newcommand{\hh}{\hookleftarrow}
\newcommand{\bD}{\textbf{D}}
\newcommand{\Gm}{\mathbb{G}_m}
\newcommand{\uF}{\underline{F}}
\newcommand{\sC}{\mathscr{C}}
\newcommand{\bX}{\overline{X}^{BS/\bQ}}
\newcommand{\sT}{\mathscr{T}}
\newcommand{\sW}{\mathscr{W}}
\newcommand{\sZ}{\mathscr{Z}}
\newcommand{\sP}{\mathscr{P}}

\begin{document}

\title{Closing Steinberg Symbols of the Mapping Class Group}
\author{J. H. Martel}
\date{\today}
\email{jhmartel@proton.me}
\maketitle

\begin{abstract}
This article introduces an algebraic problem we call Closing the Steinberg symbol (CS) of the mapping class group $\Gamma:=Mod(S)$ of compact hyperbolic surfaces $S=S_g$ for $g\geq 2$. Solving (CS) requires finding finite subsets $I$ of $\Gamma$ which satisfy a subset sum condition, namely that the translates $\sum_{\phi \in I}\phi.\sB$ of a certain chain sum $\sB:=\sum_i \alpha_i$ represent a nontrivial homology cycle (see \S\ref{CS} for details). When formal solutions $I$ to (CS) satisfy additional metric convexity properties relative to a geometric model $X:=\sT_g$ of Teichmueller space, then we obtain an interesting supply of candidate equivariant deformation retracts $X \leadsto \sZ$ onto subvarieties $\sZ \hookrightarrow X$. The construction of these retracts and their properties is based on the author's Reduction-to-Singularity method \cite{martel} and \cite{martel2022}. Thus we examine the problem of Closing Steinberg symbols as an important tool for constructing small dimensional $E\Gamma$ models and candidate spines. 
\end{abstract}


\tableofcontents




%The subvariety $\sZ=\sZ_\pi$ is a closed subset of the singularity locus of an optimal semicoupling $\pi=\pi(c,\sigma, \tau)$ with respect to a source $\sigma$, target $\tau$, and cost $c$ of transporting unit source to unit target mass. Positive solutions of (CS) lead to candidate equivariant spines $\sZ$ of $\sT$ based on the Reduction-to-Singularity method of the author's thesis \cite{martel}. The deformation retracts are constructed by an auxiliary function $\eta_{avg}(x,y_0, y_1)$ representing an averaged interaction potential. [INCOMPLETE]

%The bridge from solutions to (CS) to deformation retracts $\sT_g \leadsto \sZ$ is based on the ``reduction to singularity" methods of the author's thesis \cite{martel}, where we show the homotopy type of $\sZ$ is determined by an analytic criteron which we call Uniform Halfspace Condition (UHS). This requires an averaged vector potential denoted $\eta_{avg}(x,y_0,y_1)$ to be bounded away from zero uniformly with the target variables. Verifying (UHS) is a computational problem, and beyond the scope of this present article. The present article concludes with a question to [ref] concerning further computational problems in genus $g=2$.



%In general the retracts $\sZ$ are /`a priori of positive codimension $(>0)$. 

%And equivariant retracts onto codimension one hypersurfaces is widely known in literature [ref]. But our method can readily produce larger codimension, even $codim_{\sT} \sZ > > 2$ when the target $\tau$ exhibits sufficient symmetries. 


\section{$Mod(S)$ and Bieri-Eckmann Duality}

We let $S=S_g$ be a closed hyperbolic surface of genus $g \geq 2$, and let $\Gamma:=Mod(S)$ be the mapping class group of $S$. The topologists define $$Mod(S):=\pi_0(Diff_+(S))$$ as the group of orientation preserving diffeomorphisms of $S$ modulo isotopy. The algebraists define $Mod(S)$ via Dehn-Nielsen-Baer's theorem $$Mod(S)=Out(\pi_1(S))$$ where $\pi_1(S)=\pi_1(S, pt)$ is Poincar\'e's pointed fundamental group, see \cite{primer}.
 
 
 

%See [ref] for proofs. %For computations we find neither representation sufficiently effective, e.g. [ss]. 




%However the results described apply more generally to discrete groups $\Gamma$ which satisfy Bieri-Eckmann's homological duality and having finite virtual cohomological dimension. [Ref].

%[Intro]
%This article studies some homological properties of $\Gamma$, and their relation to constructing small dimensional $E\Gamma$ models (so called ``spines"). Specifically we introduce a problem called ``Closing the Steinberg symbol" and explain its application to constructing spines. The basic ideas were developed in our PhD thesis [ref].

%Naively, it's clear from experience that precise fundamental domains (precise reduction theory) is essentially an intractable problem. Moreover it's not clear that a precise fundamental domain is even desirable, or useful, in constructing spines. Our thesis [ref] introduces a general method of ``reduction to singularity", by which spines are represented as the singularity locus of an optimal transportation. We elaborate below.

The group-theoretic (co)homology of $\Gamma$ is defined via the symmetries of proper discontinuous actions $\Gamma \times X \to X$ on $E\Gamma$ models $X$. There is extensive literature on this subject, c.f.\cite{Brown}. The standard $E\Gamma$ model for the mapping class group is Teichmueller's $X=\sT_g$, a topological $(6g-6)$-cell $\simeq \bR^{6g-6}$, e.g. \cite{hubbard}. We assume here the basic facts that $\Gamma$ acts proper discontinuously on $\sT_g$ with finite covolume. Of course $\sT$ is a cell by Teichmuller's theorem and therefore topological trivial. However the $\Gamma$-equivariant topology of $\sT$ is highly nontrivial.  %There are alternative models which are also worth exploring, as will be discussed below \S \ref{s2}. 

%There is further a canonical uniform source measure on $\sT_g$

It is a fundamental observation of Harvey \cite{Harvey}, Harer \cite{Harer1986}, Ivanov \cite{ivanov2015virtual}, that $\Gamma$ is a Bieri-Eckmann virtual duality group \cite{BiEck}. Here a key role is played by the action of $\Gamma$ on the simplicial curve complex $\sC$ and its reduced singular homology and chain groups. Recall the curve complex $\sC=\sC(S)$ of the surface is the simplicial complex whose $0$-skeleton (vertices) $\sC^0$ consists of simple closed curves (modulo isotopy), and where a simplex exists between vertices $a,b,c,\ldots$ if the curves are simultaneously pairwise disjoint on $S$. For example in genus $g=2$, the curve complex $\sC(S_2)$ is a two-dimensional infinite simplicial complex. Obviously $\Gamma$ acts on $\sC$, c.f. \cite{Broaddus2012}. Now we present the formal definition of homological duality.

%[Below seems premature: invokes dualizing module \bD before definition]
% See section [ss] below.

\begin{dfn}
\label{hd}
A finitely generated group $\Gamma$ is a duality group of dimension $\nu \geq 0$ with respect to a $\bZ \Gamma$-module $\bD$, if there exists an element $[B]\in H_\nu(\Gamma; \bD)$ with the following property: for every $\bZ \Gamma$-module $A$, the ``cap-product with $[e]$" defines $\bZ \Gamma$-module isomorphisms $H^d(\Gamma;A) \approx H_{\nu-d}(\Gamma; A \otimes_{\bZ \Gamma} \bD)$, $f\mapsto f\cap [e]$. 
\end{dfn}

The basic properties of Bieri-Eckmann's homological duality are summarized in the following theorem from \cite{BiEck}.
\begin{thmm}[Bieri-Eckmann]\label{dual1}
Let $\Gamma$ be duality group of dimension $\nu$, with dualizing module $\bD$. Then 

(i) we have $\bZ \Gamma$-isomorphism $\bD \approx H^\nu(\Gamma;\bZ \Gamma) \neq 0$, so $\bD$ is a torsion-free additive abelian group;

%\item[(ii)] the dualizing module $D$ is finitely-generated $\bZ\Gamma$-module;

(ii) the homology group $H_\nu (\Gamma; \bD)$ is infinite cyclic generated by $[e]$ as additive abelian group; 

(iii) the group $\Gamma$ has cohomological dimension $cd(\Gamma)$ equal to $\nu$.
\end{thmm}


\begin{proof} 
The statements are direct consequences of Definition \ref{hd}. (i) We see $H^\nu(\Gamma; \bZ \Gamma)$ $\approx H_0(\Gamma; \bD)\approx \textbf{D}$. (ii) Duality implies $H^0(\Gamma; \underline{\bZ} )$ is isomorphic to $H_\nu(\Gamma; \underline{\bZ} \otimes_{\bZ \Gamma} \bD)$, which in turn is canonically isomorphic to $H_\nu(\Gamma; \bD)$ since $\underline{\bZ}\otimes_{\bZ \Gamma} \bZ \Gamma$ $\approx \bD$. But $H^0(\Gamma, \underline{\bZ})$ is canonically isomorphic to $\underline{\bZ}$. (iii) The duality isomorphism implies for every $\bZ \Gamma$-module $A$ that $H^*(\Gamma; A)$ is isomorphic to $H_{\nu-*}(\Gamma; A \otimes \bD)$ which reduces to $0$ whenever $\nu-*<0$.
\end{proof}

Most importantly for the mapping class group $\Gamma$, the dualizing module $\bD$ can be identified with the reduced homology of the simplicial curve complex. That is, we can identify $\bD$ up to $\bZ\Gamma$-isomorphism as $\bD=\tilde{H}_*(\sC;\bZ)$, where the reduced homology group inherits the natural structure of $\bZ \Gamma$-module from the action of $\Gamma$ on $\sC$. The curve complex has the homotopy-type of a countable bouquet of $(2g-2)$-dimensional spheres, c.f. \cite{ivanov2015virtual}, \cite{Harer1986}. Thus we deduce that $$vcd(\Gamma)=6g-6-(2g-2)+1=4g-5.$$ For example in genus $g=2$, this implies that $\sT_2$ is a $6$-dimensional $E\Gamma$ model, and the quotient $\Gamma \backslash \sT_2$ has the topology of a $3$-dimensional complex. Thus arises the problem of constructing spines or souls for $\Gamma$ and we seek explicit constructions or realizations of this $3$-dimensional singular subvariety for genus $g=2$. Constructing such subvarieties is the purpose of our (CS) program, as we describe below.

The problem of (CS) is based on homological properties of $\Gamma$, and specifically a representation of Bieri-Eckmann's dualizing module $\bD$ with the reduced homology of an excision boundary $\bD\approx \tilde{H}_*(\del \sT[t]; \bZ)$, where $\sT[t]$ is a maximal $\Gamma$-rational excision of Teichmueller space for a sufficiently small parameter $t$. We elaborate below.

\section{Canonical Riemannian metrics on $\sT_g$ and Flat Filling}\label{s2}
In this section we describe geometric properties which are required on $\sT$ for our constructions to be effective. The difficulty is that there is no canonical Riemannian metric on $\sT$, although there is a canonical symplectic form (c.f. Wolpert, Kerkhoff, et. al). The construction of $\Gamma$ equivariant metrics $d$ on $\sT$ is non canonical and there are many candidates. Popular metrics are Teichmueller's original metric $d_{Teich}$ \cite{hubbard},  Weil-Peterson's metric $d_{WP}$ \cite{hubbard}, Thurston's metric \cite{wolpert1986a}, or McMullen's \cite{mcmullen2000moduli}. 

For a choice of invariant Riemannian metric $d$ on $\sT$ and invariant function $$t: \sC^0 \to \bR_{>0}$$ we define an excision $\sT[t]$ of $\sT$ by excising (``scooping out") horoballs centred at various $\Gamma$-rational points at-infinity $\lambda$ and with radius $t(\lambda)$. We focus on these $\Gamma$-rational horoballs of $\sT$ because they have $\Gamma$-invariant boundary horospheres. Thus $\Gamma$ acts proper discontinuously on both $\sT[t]$ and $\del \sT[t]$, and we obtain the important diagonal action $$\Gamma \times \sT[t] \times \del \sT[t] \to \sT[t] \times \del \sT[t]. $$ We further emphasize those parameters $t$ which are sufficiently small, in which case the excisions $\sT[t]$  have a $\Gamma$-equivariant topological boundary $\del \sT[t]$ with the homotopy-type of $\sC$. This important fact implies a canonical equivariant isomorphism \begin{equation}\label{caniso}
\bD\approx \tilde{H}_*(\del \sT[t]; \bZ).
\end{equation} For the constructive topologist the $\Gamma$-action on $\sT$ is more accesible than the abstract algebraic action on $\bD$. The homologically essential spheres of $\sC$ can be viewed as spheres at-infinity within the excision $\sT[t]$. The $\Gamma$-orbit of these spheres and their singular chain sums generates an important topological $\bZ \Gamma$-module called the \emph{Steinberg module}. Following convention we designate the generator of this module a Steinberg symbol $B$. For further references to Steinberg (``modular") symbols, we refer the reader to \cite{manin}, \cite{AR}, \cite{AGM}, \cite{Stein}, \cite{Sol} and references therein. 

The contractibility of $\sT[t]$, $\sT$, and the long exact sequence in relative homology implies the natural boundary morphism $$\delta: C_*(\sT[t], \del \sT[t]) \to C_{*-1}(\del \sT[t])$$ is an isomorphism. Here $C_*$ denotes the singular chain groups. However what we require for applications is an inverse operation which is well-defined directly on singular chains, namely $$FILL:=\delta^{-1}: C_{*-1}(\del \sT[t]) \to C_*(\sT[t], \del \sT[t]).$$ This inverse operation is a \emph{filling} operation and requires a choice of metric $d'$. For our applications, any metric $d'$ with the following properties is sufficient:

\begin{itemize}
\item[(M1)] the metric $d'$ is metrically complete and proper in the interior of $\sT$;
\item[(M2)] the metric has nonpositive sectional curvature $(\kappa \leq 0)$ in $\sT$;
\item[(M3)] the $(2g-2)$-dimensional spheres generating the Steinberg symbol at infinity admit unique $d'$-flat fillings $(\kappa =0)$ to relative cycles in $\sT[t]$ mod $\del \sT[t]$.
\end{itemize}

If we examine the usual metrics, we find the WP metric has properties (M1), (M2). We observe that (M2) has the important consequence that WP-horoballs are geodesically convex in $(\sT, d_{WP})$, c.f. \cite{GroCurv}. With respect to Teichmuller's original metric, we know (M1) holds, but (M2) fails and Teichmueller's original metric has regions of positive curvature. Moreover recent work of \cite{RafiBourque} shows that convex hull constructions are not possible in Teichmueller's metric. The work of S. Wolpert implies that $d_{WP}$ also satisfies (M3). 

%% Note: the following lemma does not establish M3 because the curves in the pants decomposition do not generate the sphere, but rather a face of the sphere, i.e. a filled simplex in the curve complex.

\begin{lem}[Wolpert]
Let $Q$ be a geodesic pants decomposition of the hyperbolic surface $(S,g)$. Let $\mathscr{Q}$ be the submanifold of $\sT$ passing through $(S,g)$, and such that the Nielsen twist tangents $t(a)$ vanish for every geodesic curve $a \in Q$. Then $\mathscr{Q}$ is a totally flat submanifold of $\sT$ having vanishing sectional curvatures with respect to the WP metric $d_{WP}$. 
\end{lem}
\begin{proof}

\end{proof}

Equivalently we find $\mathscr{Q}$ consists of all hyperbolic surfaces such that the dual geodesic pants $Q^*$ intersect $Q$ orthogonally. This is equivalent to saying $\mathscr{Q}$ consists of all surfaces obtained from $(S,g)$ by varying the length parameters $\ell_a$ ($a\in Q$) and fixing all twists equal to zero in Fenchel-Nielsen coordinates associated to $Q$. 

% The author does not know if (M3) holds, although \cite{FarbRank} appears to suggest otherwise. 

Any metric $d$ satisfying the properties (M123) on $\sT$ readily leads to the construction of equivariant homotopy reductions of $\sT$ onto candidate spines $\sZ$ according to \cite{martel}. Specifically when (M3) holds, the Steinberg symbol $B$ canonically fills to a flat relative cycle $(P, \del P)$ in $(\sT[t], \del \sT[t])$, and these flat relative cycles $P=FILL[B]$ and their $\Gamma$-translates are called ``panels". The motivation for the terminology is given in \S \ref{CS} below. The panels are homologically nontrivial relative cycles in $\sT[t]$ modulo the boundary $\del \sT[t]$. In practice the condition (M3) could be slightly weakened since the application of our reduction to singularity method \cite{martel} does not strictly need the Steinberg symbols to admit flat-fillings -- what's essential is the geometric uniqueness of the fillings.

The following is important lemma for our method. 

\begin{lem}\label{kl}
Let $d$ be a metric on $\sT$ satisfying properties $(M123)$. Let $\sT[t]$ be a $\Gamma$-rational excision with sufficiently small parameter $t$ such that the canonical isomorphism \eqref{caniso} holds. Let $B$ be a Steinberg symbol with flat-filling $P=FILL[B]$. Then $P$ has zero geometric self-intersection in the quotient $\Gamma \backslash \sT[t]$, and the quotient projection $\sT[t] \to \Gamma \backslash \sT[t]$ maps $P$ isometrically onto its image.
\end{lem}
\begin{proof}
% By symmetry we argue that the principle curvatures of $P$ vanish. This implies $P$ is flat?
\end{proof}

To motivate Lemma \ref{kl}, recall that if $S$ is a closed hyperbolic surface and $\alpha$ is a closed geodesic on $S$, then the lifts $\tilde{\alpha}$ of $\alpha$ to the universal covering $\tilde{S}$ form a $\pi_1(S)$ orbit in $\tilde{S}$ where all the translates are disjoint. Likewise Lemma \ref{kl} asserts that the $\Gamma$-translates of $P$ are \emph{disjoint} in the interior of $\sT[t]$. 

By contrast the existence of parabolic elements $\gamma$ in $\Gamma$ shows that the relative cycle $P$ and its parabolic translates $\gamma.P$ intersect asymptotically ``at infinity" when $t\to 0^+$. However there remains no self-intersection in the \emph{interior} of $\sT[t]$.

%Wolpert [ref] has important formulas describing gradients of WP distance function. 

%Thus the property (b) in [ss] is crucial, as we follow Gromov's ``filling" arguments from [ref: VBC].




\section{Closing Steinberg Symbols: Definition and Properties}\label{CS}
The problem of Closing Steinberg is informally related to stitching a closed football $F$ from a sequence of panels $\{P_i\}_{i\in I}$. The panels $P_i$ are required to have the property that $F=conv\{P_i~|~i\in I\}$ and such that $\sum_{i\in I} \del P_i=0$ over $\bZ/2\bZ$-coefficients. In otherwords the problem requires finding a sequence of panels $P_i$ ($i\in I$) which assemble to a closed compact convex subset $F$ as defined above. The panels $P=P_i$ of the above footballs are analogous to the flat-filled Steinberg symbols $P=FILL[B]$ and their translates $\gamma.B$ ($\gamma\in \Gamma$). Compare Figure \ref{nikeball}.

 %where $\bD:=\tilde{H}_\nu (\sT[t]; \del \sT[t])$ is the reduced singular homology group of the excision model $\sT[t]$ constructed above, and $\underline{\bZ}/2$ is the 

%The integer $\nu$ designates the (virtual) cohomological dimension of $\Gamma$.  Bieri-Eckmann's homological duality implies $$H_0(\Gamma; \underline{\bZ}_2 \Gamma \otimes \bD) \approx H^\nu(\Gamma; \underline{\bZ}_2 \Gamma)\approx \underline{\bZ}_2 \otimes \bD $$ where the inverse isomorphism is realized by cap-product with a ``fundamental-class" element $[B]\in H_\nu(\Gamma; \bD)$. 

\begin{figure}
\centering
\includegraphics[width=0.6\textwidth]{ball-nike-pitch.jpg}
\caption{Isometric translates of hexagonal and pentagonal panels $P_i$, $P_j$, $P_k$, etc., assemble to closed balls.}
\label{nikeball}
\end{figure}

Now we present the formal definition of (CS) as derived from Bieri-Eckmann's homological duality \cite{BiEck}, \cite{BS}. The key algebraic construction is the definition of homology with coefficients in a chain complex, \cite{Brown}, where the problem of (CS) amounts to constructing a nontrivial $0$-cycle $$\xi \in H_0(\Gamma; \underline{\bZ}_2 \Gamma \otimes \bD).$$ Here the tensor product $\otimes$ is in the category of $\bZ \Gamma$ modules. If $\Gamma$ is a Bieri-Eckmann duality group, then we find isomorphisms $$H_0(\Gamma; \underline{\bZ}_2 \Gamma \otimes \bD) \approx H^\nu(\Gamma; \underline{\bZ}_2 \Gamma)\approx \underline{\bZ}_2 \otimes_\bZ \textbf{D}\neq 0.$$ This fact implies the formal existence of nontrivial $0$-cycles. 

%Thus we deduce the formal existence of nontrivial $0$-cycles. %However our applications will require nontrivial $0$-cycles which satisfy further ``convex" assumptions, c.f. \eqref{cs} below.


The group $\Gamma$ of symmetries flips, rotates, and translates the base cycle $[P]$ throughout the space, and every finite subset $I$ of $\Gamma$ produces a finite chain sum $$\sum_{\gamma\in I} \gamma.[P],$$ with total chain boundary $$\del(\sum_{\gamma\in I} \gamma.[P])=\sum_{\gamma\in I} \gamma.\del [P].$$ The basic problem of Closing Steinberg is to produce a finite subset $I \subset \Gamma$ for which the boundary of the \emph{nontrivial} chain sum $\sum_{\gamma\in I} \gamma.[P]$ vanishes in the mod $2$ homology group. The complete definition of Closing Steinberg includes further geometric conditions on the $\Gamma$-translates $\Gamma.F$ of the the closed convex hull $F=conv[P.I]$ of the translates $B.I$. Let $\sT[t], \del \sT[t]$ be a $\Gamma$-invariant excision of $\sT$. Let $[P]$ be a flat-filled relative cycle representing a nonzero generator of $H_{q+1}(\sT[t], \del \sT[t]; \bZ)$.

\begin{dfn}[\textbf{Closing Steinberg}]\label{cs}
A finite subset $I$ of $\Gamma$ successfully Closes Steinberg if: 

\textbf{(i. nontrivial mod $2$)} the chain $\xi=\sum_{\gamma\in I} \gamma.P $ is nonvanishing over $\bZ/2$ coefficients in the chain group $C_{q+1}(\sT[t],\del \sT[t];\bZ/2)$;  

\textbf{(ii. vanishing boundary mod $2$)} the boundary $\del \xi=\sum_{\gamma\in I} \gamma.\del [P] $
vanishes over $\bZ/2$-coefficients in the homology group $[\del \xi]=0$ in $H_q(\del \sT[t]; \bZ)$;

\textbf{(iii. well-defined geometric convex hull)} the boundary-chain representing $\del \xi$ is simultaneously visible from at least one interior point $x$ in $\sT[t]$;

\textbf{(iv. well-separated gates)} there exists a finite-index subgroup $\Gamma' < \Gamma$ such that the chain sum $\uF=\sum_{\gamma \in \Gamma'}\gamma.F$ has nonempty \emph{well-separated gates} precisely equal to the principal orbit $\{\gamma.P ~|~ \gamma\in \Gamma'\}$. 
%the translates $B.\gamma$, per $\gamma\in I$, are geometrically disjoint in $X[t]$.
\end{dfn}

Our definition of Closing Steinberg was inspired by the author's study of \cite{Cremona}. In Cremona's terminology, the problem is to determine a ``relation ideal $\mathscr{R}$" and construct a ``basic polyhedron $P$ whose transforms fill the space", c.f.\cite[pp.290]{Cremona}. 


 %Cremona successfully Closes Steinberg in several cases for $\Gamma=GL(\mathscr{O}_{\sqrt{-d}})$, where $\mathscr{O}_{\sqrt{-d}}$ is the ring of integers of some Euclidean complex quadratic fields. 
The hypotheses (i)--(ii) basically require the chain sum $\xi$ to be nonzero mod $2$. The hypotheses (iii)--(iv) are convexity assumptions which need be verified for any nonzero chain. The hypothesis of well-separated gates is related to the following fact: the translates $P, \gamma.P$, for $\gamma\in \Gamma$, are either identical or geometrically disjoint in $\sT[t]$ according to Lemma \ref{kl}. However the translates $P, \gamma.P$ may have nontrivial intersection at infinity in the initial Teichmueller space $\sT$. In fact the problem of (CS) is precisly to find such nontrivial intersections at infinity, although again the intersections are disjoint in the interior of $\sT[t]$.

Obviously the group structure of $\Gamma$ allows us to restrict ourselves to subsets $I$ containing the identity mapping class $Id\in \Gamma$. In practice, formal solutions to (CS) can often be found among the torsion elements and finite subgroups of $\Gamma$, c.f. \cite{Cremona}.


\begin{prop}\label{noobs}
Let $\Gamma$ be a Bieri-Eckmann duality group with dualizing module $\bD$. Then there exists finite subsets $I$ in $\Gamma$ for which $\xi=\sum_{\gamma\in I}\gamma.P$ lies in the kernel of $\del_0$ over $\bZ/2$.
\end{prop}
\begin{proof}
The argument is homological. We interpret $\xi$ as a chain sum representing a $0$-cycle in $H_0(\Gamma; \bZ/2 \Gamma \otimes_{\bZ \Gamma} \bD)$. The hypotheses of Closing Steinberg imply $\xi$ is homologically nontrivial cycle. Bieri-Eckmann duality (Proposition \ref{dual1}) implies the kernel $\ker \del_0$ is naturally isomorphic to the induced $\bZ \Gamma$-module $\bZ /2 \otimes_\bZ \textbf{D}$ which is nonzero. 
\end{proof}

To illustrate, let the reader observe that a typical element $\phi\in Mod$ will \emph{totally displace} the curves in $\sB$ such that $\sB \cap \phi.\sB=\emptyset$ for almost every $\phi\in Mod$. On the other hand, if $\phi'$ permutes the curves such that $\phi.\sB=\sB$, then $I=\{Id,\phi'\}$ would be a solution of \eqref{bb}. However we consider this solution to be trivial in the following sense: the formal sum $[\sB]+\phi'.[\sB]=2[\sB]=0$ is itself vanishing mod $2$. Such trivial solutions are avoided in the case of higher genus closed surfaces as the work of \cite{birman2013finite} demonstrates, i.e. the identity element is the only mapping class which permutes $\sB$. Obviously parabolic type elements $\phi$, i.e. curve stabilizers $\phi\in Mod_\gamma$ satisfy$\sB \cap \phi.\sB \supset \{\gamma\}$. So naturally one is tempted to find formal solutions to (CS) by choosing a suitable sequence of parabolics. 






Our hypotheses regarding Closing Steinberg have useful consequences, which we summarize in the following theorem.
\begin{thm}\label{cs1} 
Suppose $I \subset \Gamma$ successfully Closes Steinberg (Definition \ref{cs}). Define $F:=conv[I.P]$. Then

(i) the $\Gamma$-translates $\gamma.F$ ($\gamma \in \Gamma$) form a chain sum $$\uF:= \cdots \gamma.[F] + \gamma'.[F] + \gamma''.[F]+\cdots, $$ and there exists finite-index subgroup $\Gamma' <\Gamma$ which acts as additive shift-operator on the summands of $\uF$; and

(ii) the support of the chain sum $\uF$ is a simply-connected subset of $X$, and $\uF$ is a cubical $E\Gamma'$ model.

\end{thm}
\begin{proof}
We can replace $\Gamma$ with a finite-index torsion-free subgroup $\Gamma'$ to ensure $\Gamma'$ acts freely on $X$, and therefore the diagonal action is free on $X[t] \times \del X[t]$. Moreover we can ensure $\Gamma'$ translates the flat-filled relative cycle $\gamma.[P]$, for $\gamma\in \Gamma'$ freely. Then $\gamma.[P] \neq [P]$ when $\gamma\neq Id$. The definition of Closing Steinberg implies distinct translates $F, F'$ are disjoint unless they intersect in a gate $G'=\gamma'.P$ for some $\gamma' \in \Gamma'$. So $\gamma.F$ equals $F$ only if $\gamma=Id$ is trivial. This proves the summands $\{\gamma.F~|~ \gamma\in \Gamma'\}$ of $\uF$ form a principal $\Gamma'$-set, and establishes (i). The existence of an interior point $x\in F$ which is simultaneously visible to the translates $P.I$ in $X[t]$ proves $F=conv[P.I]$ is a compact convex set, and homeomorphic to some cube. Thus $\uF$ is a chain sum of cubes, hence a cubical chain sum and therefore (ii). 
\end{proof}

%The above chain sum $\uF$ provides convenient global coordinate system on an open domain of $X$. In practice it allows us to parameterize the interior by boundary data, namely by the distances of an interior point to the boundary walls of the excision $\del X[t]$.












%As developed by Ivanov (inspired by Borel-Serre [ref]) we can also identify the dualizing module $\bD$ with the reduced homology of the rational boundary at-infinity: $\bD \simeq \tilde{H}_*(\del_{\infty, rational} T)$. With this observation 

%\subsection{}
%[Curve Complex][Broaddus, duality][Steinberg module]


%(Action on Steinberg Symbols) The curve complex consists of simple closed curves $\gamma$ on $\Sigma$. The free homotopy class $[\gamma]$ corresponds to a conjugacy class in $\pi_1$. Therefore the curve complex consists of a special subset of ``simple" conjugacy classes $[\gamma]$. The algebraic characterization of those conjugacy classes which represent simple curves appears difficult problem [ref]. Obviously the mapping classes $\Gamma$ act by translation on the conjugacy classes, and coincides to the action of $\Gamma$ on $CC$ (simple closed curves). 
%Question: How to symbolically represent the conjugacy classes? Given a faithful linear representation $\rho$ of $\pi$, how to represent conjugacy classes? 
%Problem: Want to formally (CS) on computer. How to find translates of the Steinberg symbol? [Aramayona-Leininger, Birman-Broaddus-Leininger]










%\subsection{Closing Steinberg (CS)}
%The definition of (CS) is a motivated by the standard idea of fundamental domains, \cite[\S 3.5, pp.163]{PR}. Let $(X,d,\sigma)$ be a mm-space, with isometric action $X\times \Gamma \to X$. A \textit{fundamental domain} of $(X,\Gamma)$ is a measurable subset $D\subset X$ satisfying: 

%(FD1) the union of the $\Gamma$-translates covers $X$, so $\cup_{\gamma \in \Gamma} D.\gamma = X$; 

%(FD2) the translates have null intersection $\sigma[D \cap D\gamma]=0$ for every $\gamma\neq id_\Gamma$; and

%(FD3) the measure $\sigma[D]$ equals the volume of the quotient space $X/\Gamma$ with respect to the pushforward measure $p\# \sigma$, where $p: X \to X/ \Gamma$ is the quotient map. 

%The reader should observe that explicit constructions of fundamental domains are nontrivial and frequently impossible. Defining the fundamental domain $D$ is essentially equivalent to specifying all generators and relations in the group $\Gamma$. But imperfect knowledge of $\Gamma$ restricts one's ability to construct such precise fundamental domains. Acknowledging our imperfect \`a priori knowledge of $\Gamma$, we introduced the definition of ``Closing the Steinberg symbol" in [ref]. We abbreviate this subprogram ``(CS)" for ``Closing Steinberg". (CS) is a weaker condition than constructing fundamental domains. Instead (CS) seeks a measurable subset $D$ such that a stronger form of (FD2) is satisfied and such that $\cup_{\gamma \in \Gamma} D.\gamma$ is a simply-connected subdomain of $X$, i.e. the inclusion $\cup_{\gamma \in \Gamma} D.\gamma \subset X$ is a $\Gamma$-equivariant homotopy-equivalence. So (CS) constructs a truncated fundamental-type domain.  Our constructions below construct domains $D=F[t]$ equal to the convex-excision of a compact convex subdomain $F$ of $X$. 

%Explicit constructions of fundamental domains for the action $\sT \times \Gamma \to \sT$ is practically impossible, and requires precise knowledge of generators and relations in $\Gamma$. Acknowledging our imperfect \`a priori knowledge of $\Gamma$, we introduced the definition of ``Closing the Steinberg symbol" in [ref], abbreviated (CS). 























\section{(CS) for Genus Two Mapping Class Group $(g=2)$}
To illustrate our ideas, we now study the case of genus two closed Riemann surface. The duality theory of mapping class groups $\Gamma=Mod(S_g)$ for genus $g=2$ has been described by \cite{Broaddus2012}. For reference we include the following figure taken from \cite[Fig.10]{Broaddus2012}, see \eqref{br}.%Let $B$ be Broaddus' two-sphere in the curve complex $\sC$. 

\begin{figure}\label{br}
\centering
\includegraphics[width=0.4\textwidth]{broaddus-sphere-marked-new.png}
\caption{Homologically nontrivial $2$-sphere in the curve complex $\sC$ of genus $2$ closed surface. Figure adapted from \cite[Fig.10]{Broaddus2012}}
\end{figure}
%Observe $B$ has nine vertices $\alpha_1, \alpha_2, \alpha_3,\beta_1, \beta_2, \beta_3.$ The sets of curves $\{\alpha_1, \alpha_2, \alpha_3\}$ and $\{\alpha_4, \alpha_5, \alpha_6\}$ are pants decompositions. The curves $\beta_1, \beta_2, \beta_3$ are separating curves on $\Sigma$. Then (CS) requires a finite subset $I \subset \Gamma$ such that the chain sum $$SUM [\{\alpha_i.\gamma, ~\beta_j.\gamma |~~i=1,\ldots,6,~~j=1,2,3, ~~\gamma \in I\}] $$ vanishes over $\bZ/2$-coefficients, i.e. such that $$\sum_{\gamma \in I} (\alpha_1.\gamma+\cdots+\alpha_6.\gamma+\beta_1.\gamma+ \beta_2.\gamma+ \beta_3.\gamma )=0 ~~ (mod~ 2).$$ 

%We remark that Broaddus' nine curves do not satisfy the $9g-9$ theorem, i.e. the lengths of Broaddus' curves do not sufficiently discriminate between marked genus two Riemann surfaces. 


%\subsection{Formal Solutions of (CS) in genus two}
The formal problem of (CS) for genus two surfaces has the following symbolic setup. Let $V:=\bZ/2(\sC^0)$ be the abelian topological group consisting of finitely-supported $\bZ/2$-valued functions $f: \sC^0 \to \bZ/2$ on the set $\sC^0$ of free homotopy classes of simple closed curves on a surface $S$. We abbreviate such a function $f$ with its support $\alpha+\beta+\ldots$. On the genus two closed surface, consider Broaddus' set of nine curves $\alpha_i, \beta_j, \gamma_k$ for $i,j,k \in \{1,2,3\}$, and the formal sum 
\begin{equation}\label{bb}
\sB:= \sum_{i,j,k=1}^{3} \alpha_i+\beta_j+\gamma_k.
\end{equation}

Now finally we make the problem of (CS) totally explicit: 
\begin{dfn}
A finite subset $I\subset \Gamma$ formally Closes Steinberg for the mapping class group $\Gamma$ of genus two closed surfaces if 
\begin{equation}
\sum_{\phi\in I} \sum_{\alpha\in \sB}\phi.\alpha=0, ~~mod~2
\end{equation}

where the zero element $0$ on the right hand side is the zero element in $V$, i.e. the constant zero-valued distribution on $\sC^0$. 
\end{dfn}

Symbolically the ``vanishing mod $2$" of the translates $\sum_{\phi\in I} \phi.\sB$ says there is an even number of coincidences between the translated curves $\phi.\alpha$ where $\phi\in I$, $\alpha\in \sB$. This can be implemented on python by iterated symmetric differences. For example if $\sB$ denotes the \emph{set} of curves defined in \eqref{bb}, then a finite subset $I=\{\phi_1, \ldots, \phi_n\}$ is formal solution to (CS) if and only if $$\phi_1. \sB \Delta \cdots \Delta \phi_n.\sB = \emptyset.$$ We can omit the parentheses since the symmetric difference $\Delta$ is an associative operator.

Now we introduce some standard notation following \cite{nakamura2018generation}. Let $$\eta=aecf$$ be the order ten element in $Mod(S_2)$ and define $$\mu:=\eta^4.$$ Then $\mu$ is an order five element in $Mod$. If $a,b,c$ is a geodesic pant decomposition, then we define the chain sum $$B:=[a]+[b]+[c]+\mu.[a]+\mu.[b]+\mu.[c].$$

\begin{lem}
Let $I_0:=\{Id, \mu, \mu^2, \mu^3, \mu^4\}$. Then $\sum_{\phi \in I_0} \phi.B=0$ (mod $2$) and $I_0$ is a formal solution to (CS). 
\end{lem}
\begin{proof}
The vanishing of the chain sum $\sum_{\phi \in I_0} \phi.B$ is clear. Moreover all the summands $\phi.B$ are distinct for $\phi \in I_0$ and this proves the formal solution is nontrivial.
\end{proof}

By computation using Mark C. Bell's curver [ref] we have found that the $I_0$-translates of $B$ is supported on ten curves. Given the formal solution $I_0$, we proceed in several steps. First we construct  the convex hull $$F:=conv( I_0.B )$$ over these ten curves constituting the $I_0$ translates of $B$. Then we need estbalish that the chain sum $$\underline{F}:=\sum_{\phi \in \Gamma} \gamma.F$$ has a \emph{well-separated gates structure} equal to $\Gamma.B$. The idea of well-separated gates is introduced in \cite[pp.13, \S 5.1]{martel}, and means $\uF=\sum_{i\in I} F_i$ is a countable chain sum of sets $F_i$ where the intersections $G:=F_{ij}:=F_i\cap F_j$ form a principal $\Gamma$-set.

%To verify that $\uF$ has well-separated gates using Bell's curver program: [insert].
%[Caution: Do we need be more careful about finite-index subgroups $\Gamma'$ and $\Gamma$?]





\section{}











 


%Thus we are forced to search out solutions to (CS) directly. While \eqref{bb} provides an explicit symbolic statement of (CS), it remains difficult to actually search for such solutions. The problem is to represent the action of the mapping class group on the simplicial curve complex $\sC$. The curve complex $\sC$ consists of free homotopy classes of simple closed curves $\alpha$ on $S$. The free homotopy class $[\alpha]$ corresponds to a conjugacy class in $\pi_1$. Therefore the curve complex consists of a special subset of ``simple" conjugacy classes $[\alpha]$. The algebraic characterization of those conjugacy classes which represent simple curves appears difficult problem.








%\section{}
%We remark that \emph{if} the mapping class group admitted an effective linear representation, then our methods for solving (CS) for arithmetic groups could be extended. For the basic arithmetic group $\Gamma=PGL(\bZ^2)$, a convenient solution to (CS) is given by $$I_0=\{Id, \begin{pmatrix} 1 & 1 \\ 0 & 1 \end{pmatrix}, \begin{pmatrix} 0 & 1 \\ -1 & 2 \end{pmatrix}\}.$$ Thus the chain sum $$\xi:= 
%(\begin{bmatrix} 1 \\ 0 \end{bmatrix} \otimes \begin{bmatrix} 0 \\ 1 \end{bmatrix}) + 
%(\begin{bmatrix} 0 \\ 1 \end{bmatrix} \otimes \begin{bmatrix} 1 \\ 1 \end{bmatrix}) +
%(\begin{bmatrix} 1 \\ 1 \end{bmatrix} \otimes \begin{bmatrix} 1 \\ 0 \end{bmatrix}) $$ represents a nontrivial $0$-cycle $\xi \in H_0(PGL(\bZ^2) , \bZ_2 \otimes \textbf{D})$, where $\textbf{D}$ is the dualizing $\bZ PGL(\bZ^2)$-module $\approx H_1(Proj[\del Q[t]]; \bZ)$. So $\del_0 \xi=0$ where $\del_0$ is the chain boundary operator. The reader can consult \cite[]{martel}. 



%Indeed the problem of (CS) has the following form: we look for subsets $\{\alpha'\}$ of $\sC$ and want nontrivial elements $\phi\in MCG$, $\phi\neq Id$, such that the intersection $\{\alpha'\}' \cap \{\alpha'.\phi\}'$ is nonempty.
















%\section{Well-Separated Gates: From (CS) To Candidate Spines}
%Suppose the user succesfully Closes the Steinberg symbol, i.e. finds a finite subset $I$ of $\Gamma$ satisfying conditions \ref{cs}. Solutions to (CS) allow us to replace $\sT$ and rational excisions $\sT[t] \subset \sT$ with \emph{ a chain sum $\uF$ with \textbf{well-separated gates}}, a term which also appears in item \eqref{cs}(iv) in the definition of (CS). 

% In our applications the chain summands $F_i$ are convex excisions of the form $F_i=F[t]$ (see \cite[\S 5.5]{martel}). The idea is next that gated costs (see \cite[\S 5.2]{martel}) are determined by their restrictions to the summands $F_i$, and also by the gates $G\subset F_i$ which are contained in the given summand. Thus we reduce the study of costs on $\sT[t]$ to localized costs defined on the summands $F_i$ and with respect to the gates $G\subset F_i$. This reduces us to the setting of \cite[\S 5.9]{martel} where we studied various repulsion costs on convex excisions. 

If $\uF$ is a chain sum with well-separated gates $\{G\}$, then the singularity locus $\sZ$ naturally decomposes as a chain sum $\sZ=\sum_i \sZ \cap F_i$, and where $\sZ \cap F_i$ is the singularity locus of a restricted semicoupling program, with respect to the restricted cost $c|F_i$. Best results are obtained with costs satisfying Properties (D0)--(D4) and we conjecture that the visibility costs satisfy (D0)--(D4) using the notation of \cite{martel}. Finally using the Reduction to Singularity method of \cite[Theorems 1.4.1-2]{martel}, we naturally construct continuous deformation retracts and which even assemble to global continuous retracts $\sT \leadsto \sZ$. 

N.B. Constructing the retraction is contingent on the user having an effective computable model of $\sT$ available. Solutions to (CS) allow us to localize all computations onto the local chain summands. Generally $\sZ$ has large codimension in $\sT$ depending on so-called Uniform Halfspace Conditions. Symmetries in the excision boundary (and target measure) on $\del \sT[t]$ increases the maximal codimension of $\sZ$ with the possibility of attaining the extreme codimension, even the equivariant spine of $\sT$. 
%\section{On Canonical Geometric $E\Gamma$ models.}

%[INCOMPLETE]

%Our thesis \cite{martel} studied the importance of using different models for various $E\Gamma$ classifying spaces. So what is the best model of Teichmueller's space $\sT$? Indeed this article has not yet formally defined the \emph{points} of $\sT$, neither have we defined the action of $\Gamma=MCG(\Sigma)$ on $\sT$. 



%The discovery of models of hyperbolic geometry in the careers of Gauss, Lobachevsky, \etc, is enormous discovery for the human mind. And especially the proper discontinuous actions of $PGL(\bZ^2)$ on the unit two dimensional disk $D^2$. The incredible formula $$z.\begin{pmatrix} a & b // c & d \end{pmatrix}:=(az+b)(cz+d)^{-1}$$ describing the group of Mobius transformations on the complex plane. It is further incredible that the Mobius group acts by conformal transformation (holomorphically) on $z$ (i.e. the above formula has no terms involving $\bar{z}$ in any power series representation). %How to make the action of $PGL(\bZ^2)$ on the disk as 
 





%(?) Observe no triple of Broaddus' curves are related by a Dehn twist, i.e. no relations of the type $\alpha_k=D_{\alpha_i}(\alpha_j)$, where $D_{\alpha_i}$ is a Dehn twist about $\alpha_i$. 
%\begin{lem}
%For every marked surface $X\in \sT_2$, the gradients $\nabla_X \ell_{\alpha_i}$, $\nabla_X \ell_{\beta_j}$ with respect to WP-metric, for $i=1,\ldots,6$, $\beta=1,2,3$ span a three-dimensional convex hull, i.e. the gradients span a four-dimensional subset of $T_X \sT$. 
%\end{lem}
%\begin{proof}
%Wolpert's formula [ref] proves the gradients of cuff lengths of a pants decomposition are linearly independant and almost orthogonal. 
%\end{proof}
%The above formulation of (CS) for Broaddus' two-sphere can be further simplified. Let $\beta_1, \beta_2, \beta_3$ be the three separating curves in Broaddus' two-sphere $B$. 
%\begin{prob}
%Determine finite subsets $I \subset \Gamma$ such that $$\sum_{\gamma \in I} \beta_1.\gamma+\beta_2.\gamma+\beta_3.\gamma =0 , ~~(mod~2).$$ 
%\end{prob}
%A solution to Problem [ref] has the form $$abc+bcd+aef+efd.$$ This requires finding $\phi, \phi' \in \Gamma$ such that $$\phi(a)=a, \phi(b)=e, \phi(c)=f$$ and $$\phi'(a)=d, \phi(b)=b, \phi(c)=c,$$ where $a,b,c,\ldots$ are separating curves.
 

%(punctured torus) Observations: if $\beta$ is a separating curve, then $\Gamma_\beta$ is isomorphic to a direct product $GL(\bZ^2) \times GL(\bZ^2)$, where we recall $MCG_+(\Sigma_{1,1}) \approx GL(\bZ^2)$. 

%[Insert: Wolpert, gradient computations]

%\section{Convexity and $9g-9$ Theorem}
%Observe that Lemma [ref].(ii) implies $T[t]$ coincides with the closed visibly-convex hull of $\del T[t]$ in $T[t]$. 
%The reader may compare this observation with Wolpert's theorem that $\bar{T}$ is the geodesic convex hull of marked maximally-noded Riemann surfaces. [ref?] 

%The augmented Teichmuller space $\bar{\sT}$ with $WP$-geometry has certain analogy to Voronoi's cone $V$ of positive-semidefinite quadratic states $q: \bR^N \to \bR_{\geq 0}$, $q(v):={}^t v Q v$ where $Q$ is a positive semidefinite symmetric operator and $v\in \bR^N$. Krein-Milman's theorem [ref] implies $\bar{V}$ is the closed convex hull of rank-one quadratic forms, specifically $\ell(v)^2$, where $\ell \in \bR^{N *}-\{0\}$ is a nontrivial linear form. [Ref]. 

%Analogous to Voronoi's state space [ref], there are infinitely-many faces meeting at the extreme points $\sE$ of $\bar{T}$. Compare [ref: Gunnell's appendix, Stein, Modular forms]. The visibility relation $V$ on $T[t] \times \del T[t]$ allows the application of Krein-Milman's theorem to the nonconvex excision model $T[t]$. 
  




%Let $\sT$ be Teichmueller space equipped with WP-metric $d=d_{WP}$. If $f: \sT \to \bR$ is a function, then $\nabla f$ denotes the gradient of $f$ defined with respect to $WP$-metric. 
%\begin{dfn}
%Let $X$ be a hyperbolic surface. We say a collection $\{\alpha_0, \alpha_1, \ldots \}$ of simple closed curves is affinely-independant if the gradients $\{\nabla \ell_{\alpha_0} X - \nabla \ell_{\alpha_i} X \}_{i\geq 1}$ are a linearly-independant subset of $T_X \sT$. 
%\end{dfn}

%We say the collection $\{\alpha_0, \alpha_1, \ldots \}$ has affine-dimension $d$ if $span\{\nabla \ell_{\alpha_0} X - \nabla \ell_{\alpha_i} X\}_{i\geq 1}$ is $d$-dimensional.





%Various spines have been conjectured. Notably W.Thurston [ref] proposed those surfaces whose systoles are filling as a spine. However this characterization does not appear to have the correct codimension in $\sT$, and his proposed flow is discontinuous. Recently Ji, Ji/Wolpert [ref] constructed a codimension-two deformation retract for all $g\geq 2$. Following an argument of S.Wolpert and H. Parlier, [Ji: Proposition 4.3] proves the set $S''$ consisting of $X\in \sT_g$ with the properties  that $\#sys_1(X)\geq 3$ and there exists $\alpha, \beta \in sys_1(X)$ such that $|\alpha \cap \beta |\geq 1$ is a codimension-two deformation retract for $g\geq 2$. Harer [ref: Harer] obtained explicit spines for punctured mapping class groups $MCG(\Sigma_{g,p})$ where $p\geq 1$ is the number of punctures. This ``punctured" case is considerably easier than the closed case $p=0$. 




%[Below: Incorrect]
%This leads us to conjecture that the set of all genus $g$ hyperbolic surfaces $X$ whose systoles $sys_1(X)$ has affine-dimension $\geq 2g-1$ forms a $MCG$-equivariant spine of $\sT(\Sigma_g)$. Thus we propose
%$$\sW_g:=\{X\in \sT_g |~\dim(span\{\nabla \ell_\alpha X | ~\alpha\in sys_1(X) \})\geq 2g\}$$ is a closed $MCG$-equivariant deformation retract of $\sT_g$. The above subset $\sW_g$ evidently has the correct codimension in $\sT$, as required by the fact that $$vcd(MGC_g)=6g-6-(2g-1)=4g-5,$$ and it remains to prove $\sW_g \hookrightarrow \sT_g$ is an equivariant deformation-retract. Alternatively we can characterize $\sW$ as consisting of hyperbolic surfaces $X$ whose systoles $sys_1(X)$ contains at least $2g$ disjoint simple closed curves. 
%The set $S''$ is evidently different than our set $\sW_g$ for $g\geq 2$. 
%In analogy with Soul\'e-Ash's well-rounded retract, we seek an elementary continuous flow which deforms $\sT$ onto $\sW$.  
%Thus we are proposing that the set of hyperbolic surfaces $X$ whose systoles $sys_1(X)$ are disjoint is a spine for $\sT$. Since the maximum number of disjoint simple closed curves on a genus $g$ surface is $2g-1$, we can equivalently characterize the spine as those surfaces admitting a ``systolic pants decomposition". 






%\section{}
%The purpose of this note is to highlight an elementary problem which we call ``Closing the Steinberg symbol", and to describe its application to constructing minimal $MCG$-invariant spines of Teichmuller's space $\sT$. We begin with the genus $g=2$ oriented closed surface. The starting point is Broaddus' $2$-sphere $B$ desribed in [ref]. This $2$-sphere is a homologically nontrivial sphere in the simplicial curve complex $\sC=\sC(\Sigma_2)$. Considered as a $\bZ MCG$-module, the reduced integral homology groups $\tilde{H}_2(\sC;\bZ)$ represents the dualizing module $D$ of $MCG$. Broaddus' $2$-sphere $B$ and its translates by $MCG$ generate the $\underline{\bZ/2\bZ} MCG$-module $D\otimes \bZ/2\bZ$. Now we imagine $B$ as the ``panel", and seek some translates $B, B.\phi_1, B.\phi_2, \ldots$ of panels whose chain sum $B+ B.\phi_1+ B.\phi_2+ \cdots$. The problem is motivated by stitching a football from panels. [ref].

%\section{Teichmueller and Weil-Peterson metrics}
%The present note constructs an explicit spine for the mapping class group $\Gamma=MCG(\Sigma_g)$ of a closed Riemann surface of genus $g\geq 2$. We construct a closed Lipschitz subvariety $\underline{Z}=Z_{J+1}$ of the Teichmueller space $T=Teich(\Sigma_g)$ such that $\underline{Z} \hookrightarrow T$ is a $\Gamma$-equivariant continuous deformation retract and where $\dim(\underline{Z})=vcd(\Gamma)$. In otherwords $\underline{Z}$ is a minimal-dimension $E\Gamma$ classifying space. Our construction depends on special finite subsets $I \subset \Gamma$ of mapping class elements which successfully ``Close the Steinberg symbol", see Definition [ref] below. 

%The constructions of equivariant deformations of Teichmueller space have typically depended directly on systole-type functionals, which are Morse-like functions, c.f. [Ji], [Ji-Wolpert], [Bavard]. However these methods only achieve codimension $\leq 2$ retracts, and are rather limited. There is famous preprint of Thurston \cite{} which contains major error and constructs a discontinuous retract. Our thesis develops a new general method for constructing souls and spines, based on Kantorovich's singularity functor $Z$, which is a contravariant functor $Z: 2^{Y} \to 2^{X}$ defined $Z=Z(v, \sigma, \tau)$ where $v:X\times Y \to \bR$ is cost function, $(X,\sigma)$ is a source mm-space and $(Y,\tau)$ is a target mm-space. 
%We apply our general method to the source space $X=T[t]$, target $Y=\del T[t]$, with visibility cost $v$, where $T[t]$ is a convex-excision of Teichmueller's space $T=Teich(\Sigma_g)$. 








%Observation: if $\{P_i\}$ is a tessalation of $X$ with vertices at finite positions in $X$, then the tiling is not necessarily aspherical, and the chain sum $\sum P_i$ will not be homotopy-equivalent to $X$. Indeed, the existence of vertices at finite positions allows the possibility that $\sum P_i$ admits nontrivial (finite) coverings. However if all vertices exist ``at-infinity", then there exists no finite lifts (and no finite coverings). 

%Thurston Straightening: an ideal simplex admits no self-mappings of degree $\geq 2$. Analogously a relative simplex $(B, \del B)$ in $(X[t], \del X[t])$ admits no self-mapping of degree $\geq 2$. 
%Lemma: If $X[t]$ is excision model as above, then $X[t] \times \del X[t]$ does not admit any self-mappings of degree $\geq 2$.   


%[Rephrase: find finite subgroup of MCG such that action on curve-complex CC(S) has orbit which `Closes Steinberg'?] Sample the Connoly-Wosneiscki paper, with explicit finite subgroups -- any accidents? 

%For case of genus two, we observe that a Dehn twist $D_\gamma$ about a vertex $\gamma$ of $P$ will fix the link of $\gamma$ pointwise, and $D_\gamma$ will therefore translate exactly three vertices $\delta_1, \delta_2, \delta_3$ of $P$ to three new vertices $ \delta'_1, \delta'_2, \delta'_3$ of $P.D_\gamma$. The chain sum $P+P.D_\gamma$ therefore has total of six vertices, after reduction modulo 2, namely $\delta_1, \delta_2, \delta_3$, $ \delta'_1, \delta'_2, \delta'_3$. Likewise, a translate of $P.D_\gamma$ by another Dehn twist $D_{\gamma'}$ [Incomplete]

%Elementary observation: we can readily identify a finite subgroup $\approx \bZ /2 \times \bZ/2$ of the genus two mapping class group. But this finite (order four) subgroup has an orbit which does not `CS'.














\printbibliography[title={References}]
\end{document}
